\documentclass{report}

\usepackage{polski}
\usepackage{tabularx}
\usepackage{booktabs}
\usepackage{amsmath}
\usepackage[margin=2.5cm]{geometry}

\title{Kryptografia z elementami algebry}

\begin{document}

\maketitle

\tableofcontents

\chapter{Wstęp do algebry} % Wykład 2022-10-10

\section{Definicja działania}

Def.
\textbf{Działaniem wewnętrznym} (dwuargumentowym) w zbiorze A nazywamy dowolne odwzorowanie $* : A \times A \to A$ co zapisujemy $ *(a, b) = a * b $.

Zero nie jest liczbą naturalną

Przykłady:
\begin{itemize}
\item Zbiór liczb naturalnych z dodawaniem $N^+ = (N, +)$
\item Zbiór liczb całkowitych z dodawaniem $Z^+ = (N, *)$
\item Zbiór liczb modulo $n$ z dodawaniem $Z_n^+$
\item Zbiór liczb $\Phi(n) = \{ a \in Z_n, \gcd(a, n) = 1 \}$
\end{itemize}

\subsection{Definicje obiektów algebry}

\begin{center}
\begin{tabularx}{\textwidth}{ c X X }
	\toprule
	\textbf{Nazwa} & \textbf{Opis} & \textbf{Przykład}  \\
	Grupoid & Dowolny niepusty zbiór z działaniem wewnętrznym & Liczby rzeczywiste z potęgowaniem \\
	Półgrupa & Grupoid w którym działanie jest łączne & Liczby dodatnie z dodwaniem \\
	Monoid & Półgrupa posiadająca element neutralny & Ciągi znaków z konkatenacją \\
	Grupa &  Monoid w którym istnieją elementy przeciwne &  \\
	Grupa abelowa & Grupa w której działanie jest przemienne & Liczby całkowite z dodwaniem \\
	\bottomrule
\end{tabularx}
\end{center}

\subsection{Własności działań}

\begin{description}
\item[Łączność] Dla $a, b \in A$ zachodzi $a * (b * c) = (a * b) * c$
\item[Przemienność] Dla $a, b \in A$ zachodzi $a * b = b * a$
\item[Element neutralny $e$] Dla $a, e \in A$ zachodzi $e * a = a * e = a$.
\item[Element przeciwny $b$] Dla $a \in A$ i elementu neutralnego $e$ zachodzi $a * b = b * a = e$. Gdy używamy oznaczeń addytywnych móiwmy \textit{przeciwny}, gdy multiplikatywnych \textit{odwrotny}.

Co ważne wprowadzając element przeciwny nie wprowadzamy nowej operacji. Liczby całkowite z dodawaniem posiadają elementy przeciwne, ale nie posiadają działania odejmowania - odejmowanie wykonujemy jako dodawanie elementu przeciwnego.
\end{description}

Element neutralny, o ile istnieje, jest wyznaczony jednoznacznie. Jeśli działanie jest łączne to każdy element posiada co najwyżej jeden element przeciwny.

\paragraph{Dowód istnienia tylko jednego elementu neutralnego}

% TODO

\paragraph{Dowód istnienia co najwyżej jednego elementu neutralnego}

% TODO

\subsection{Podgrupa}

Def.
Niepusty podzbiór $H$ grupy $(G, *)$ nazywamy podgrupą gdy dla dowolnych $a, b \in H$ element $ab^-1 \in H$ co zapisujemy $H < G$. Przykładowo: $ \Phi(8) = \{ 1, 3, 5, 7 \} $ z mnożeniem modulo 8 posiada następujące podgrupy: $\{1\}$, $\{1,3\}$, $\{1,3,5,7\}$.

Stw. Niech $G$ będzie grupą, a $H \subseteq G$ niepustym podzbiorem. Wtedy następujące warunki są równoważne:
\begin{enumerate}
\item $H$ jest podgrupą $G$
\item Element neutralny $e \in H$ oraz $H$ jest zamknięty ze wzgledu na działanie i operację odwrotności tzn. $ab^{-1} = e$ dla $a,b \in H$
\item $H$ wraz z działaniem jest grupą.
\end{enumerate}

\subsection{Warstwy}

Niech $H$ będzie podgrupą $G$. Warstwą lewostronną wyznaczoną przez element $a \in G$ nazywamy $aH = \{ ah, h \in H \}$.

Przykładowo dla $Z_8^+$ i podgrupy $H=\{0,6\}$.

\begin{align*}
0 + H = \{ 0, 6 \} \\
1 + H = \{ 1, 7 \} \\
2 + H = \{ 2, 0 \} \\
3 + H = \{ 3, 1 \} \\
4 + H = \{ 4, 2 \} \\
5 + H = \{ 5, 3 \} \\
6 + H = \{ 6, 4 \} \\
7 + H = \{ 7, 5 \}
\end{align*}

Warstwy lewostronne są albo identyczne albo rozłączne. Suma mnogościowa wszystkich wartsw lewostronnych jest równa całej grupie.

Indeksem podgrupy H w grupie G nazywamy nazywamy ilość (mnogościową) zbioru różnych wartstw lewostronnych.


Twierdzenie Lagrange: Niech G będzie grupą skończoną, a H jej podgrupą. Wtedy $|G| = (GiH)|H|$. (Można z tego wywnioskować, że rząd grupy dzieli rząd grupy).

Przykład:

Dla grupy $Z_8^+ = \{ 1, 2, 3, 4, 5, 6, 7 \}$ i podgrupy $H = \{0,4\}$ z wyżej wprowadzonej tożsamości wynika, że $|G| = (GiH)|H| \implies \frac{|G|}{|H|} = (GiH)$.

Dowód:
Pokażemy, że $|aH| = |H|$ dla $a \in G$.
Wykażemy, to poprzez to, że funkcja $f : H \to aH$ jest różnowartościowa.

\chapter{Kryptografia}

Kryptografię dzielimy na:
\begin{description}
\item[Kryptografia] Projektowanie i analiza algorytmów, tworzenie protokołów.
\item[Kryptoanaliza] Łamanie algorytmów
\end{description}

Zakładamy, że cała komunikacaj jest przesyłana otwartym kanałem, czyli formą komunikacji która jest publiczna - każdy może nasłuchiwać przesyłane komunikaty. Protokół kryptograficzny powinien być odporny na próby manipulacji zprzez jedną ze stron.

Protokół jest \textbf{bezpieczny} jeśli przez czas od jego publikacji nie zostanie znaleziona metoda na złamanie go.

Szyfrowanie doskonałe - na podstawie szyfrogramu nie mogę nic wywnioskować o wiadomości.

\section{Powszechne oznaczenia}

\begin{description}
\item[M] wiadomość
\item[C] szyfrogram
\item[K] klucz do szyfrowania
\item[E] algorytm szyfrowania
\item[D] algorytm deszyfrowania
\item[G] generator ciągów pseudolosowych
\item[$(G, E, D)$] schemat szyfrowania o następujących własnościach:
	\begin{itemize}
		\item Szyfrowanie wiadomości przy pomocy klucza daje ciąg szyfrogram $E_K(M) = C$
		\item Deszyfrowanie przy pomocy klucza szyfrogramu daje wiadomość $D_K(C) = M$
	\end{itemize}
\end{description}

Klucze powinny być losowe, tzn. każdy klucz powinien mieć takie same prawodpodobieństwo wystąpienia.

\section{Symetryczny aprotokół szyfrowania}

Cel: Bezpieczne przekazywanie informacji między stronami.

\begin{enumerate}
\item Alice i Bob uzgadaniają jawnie $(E, D)$ (algorytm szyfrowania).
\item Alice i Bob uzgadniają bezpiecznie klucz tajny $K$.
\item Alice wysyła zaszyfrowaną wiadomość (szyfrogram) $C = E_K(M)$ do Boba
\item Bob deszyfruje szyfrogram $D_K(C) = M$ i odczytuje wiadomość.
\end{enumerate}

Potencjalnym problemem tego protokołu jest: na ile bezpieczne ustalenie jest klucza, jak i na ile bezpieczny jest algorytm szyfrowania.

\section{Szyfrowanie z kluczem jednorazowym}

\begin{enumerate}
\item Alice i Bob generują klucz.
\item Alice koduje przy pomocy operacji XOR (oznaczana $\oplus$: $ E_K(M) = K \oplus M $.
\item Bob dekoduje przy pomocy operacji XOR: $D_K(C) = K \oplus C$.
\end{enumerate}

Wadą tego algorytmu jest klucz o długości wiadomości. Niestety udowodniono, że klucz który miałby osiągnąć szyfrowanie doskonałe musi mieć co najmniej tyle samo bitów co wiadomość.

\section{Bezpieczeństwo obliczeniowe}

Adwersarz nie powinien móc złamać klucza z uwagi na zbyt dużą moc obliczeniową konieczną do złamania szyfrowania jak i małym prawdopodobieństwem odgadnięcia pojedyńczego klucza.

Mówiymy, że algorytmm działający na danych, o liczbie bitów $k$ działący w czasie wielomianowym (wykładniczym jeśli liczba operacji elementarnych na bitach potrzebnych do wykonania tego algorytmu jest rzędu $O(k^c)$ ($O(e^{kc})$).

Przykładowo dodawanie dwóch liczb $a, b \in N$, takich, że $a\leq b$ ma złożoność $O(\log b)$.

\subsection{Algorytm znajdowania dzielnika}

Aby znaleźć dzielnik liczby $n$ potrzebujemy sprawdzić liczby od $3 do \sqrt n$, co daje liczbę operacji $O(e^{\frac{1}{2}\log n})$.

\chapter{Wykład 3}

% Protokół Difiiego Helmnana.
% Rachunek indeksów
% algorytm fermata, millera, AKS

Stwierdzenie.
Niech A będzie dowolnym podzbiorem grupy G. Istnieje wtedy najmniejsza w sensie inkluzji podgrupa H grupy G zawierająca A.

Podgrupę, o której mowya, nazywamy podgrupą generowaną przez zbiór A i oznaczmay $\langle A \rangle$. Zbiór A nazywamy zbiorem generatorów grupy H.

Grupę, w której istnieje skończony zbiór generatorów nazywamy skończenie generowaną.

Grupę, która posiada jendoelementowy zbiór generatorów nazywamy cykliczną.

Grupa generowana przez zbiór pusty to grupa identycznościowa.

Grupa $Z^+$ jest cykliczna, a jej generatorem jest $1$ lub $-1$.
Grupa $Z_n^+$ jest cykliczna, a jej generatorem jest 1
Grupy $\phi(2)$, $\phi(4)$ są cykliczne
Grupa $(p^k)$ jest cykliczna, gdzie $k \in P$


Rżad grupy $\langle g \rangle$ nazywanmy rzędem elementu $g$ w grupie G i oznaczamy symbolem $ord(g)$.

Dla $\phi(p)$ mamy $ord(1) = 1$, $ord(p-1) = 2$.

Dla $\phi(17)$ = $\{ 1, \dots, 16 \}$,

Jeżeli $\langle g \rangle = \phi(p^k)$, to $ord(g) = \phi(p^k)$

\subsection{Homomorfizm}

Jest to odwzorowanie $f : G \to G^-1$ jeśli $f(ab) = f(a)f(b)$.

Mówimy, że $f$ jest:

\begin{description}
	\item[Monomorfizmem], gdy $f$ jest różnowartościowa (injekcją)
	\item[Epimorfizmem], gdy $f$ jest surkiekcją
	\item[Izomorfizm], gdy $f$ jest bijekcją.
\end{description}

Gdzie dla algebraika izomorficzne grupy są identyczne, tak dla kryptografa są one różne, gdyż izomorfizm może być funkcją jednokierunkową.

\section{Złożoność obliczeniowa}

Funkcja $f : N \to R$ jest jednokierunkowa jeśli:
\begin{itemize}
\item obliczenie wartości $f$ jest czasu wielomianowego ze względu na liczbę bitów danych.
\item obliczenie wartości $f^-1$ jest czas uwykładniczego, ze względu na liczbę bitów danych.
\end{itemize}

Niech G będzie dowolną grupą skończoną. Niech $g \in G$ oraz $x < |G|$. Definiujemy $F(G, g, x) = g \in G$, oraz $F^-1(G, g, y) = x$, gdzie $y=g^x$ dla $y \in G$ o ile takie $x$ istnieje.

Dla grupy $G = Z_n^+$ można zdobyć $F^-1$ rozszerzonym algorytmem euklidesa (który nie jest wykładniczy), więc nie jest jednokierunkowa.

Dla grupy $G = \phi(n)$ można zdobyć $F^-1$ umiemy zrobić tylko w czasie wykładniczym, dlatego ta funkcja jest jednokierunkowa.

Zbiór $Ker(f) = \{ g \in G : f(g) = e' \}$ nazywamy jądrem homoformizmu $f$. Jądro jest podgrupą grupy $G$.

Zbiór $Im(f) = \{ g' \in G' : \exists_{g\in G} f(g) = g' \}$ nazywamy obrazem homomorfizmu. Obraz jest podgrupą grupy $G'$.

Twierdzenie o izomorfiźmie.

Homomorfizm którego jądro jest trywialne jest injekcją.

\chapter{Wykład 4} % 2022-11-14

\section{Problem logarytmy dyskretnego}

Dane $p, g, y_A$. Do obliczenia $x_A$ taki, że $g^{x_A} = y_A \pmod{p}$.

Na komputerze klasycznym problem wykładniczy, na kwantowym - wielomianowy.

\section{Algorytm Diffiego-Hellomana}

Dla ustalonego $p$, obliczenia wykonujemy modulo $p$.

\begin{enumerate}
\item Alice losuje $p$ oraz $g$ takie, że $g$ jest generatorem grupy $g \in \Phi(p)$
\item Alice losuje $1 < x_A < p$, Bob losuje $1 < x_B < p$
\item Alice oblicza $y_A = g^{x_A}$, Bob oblicza $y_B = g^{x_B}$
\item Alice i Bob otwartym kanałemp rzesyłają sobie $y_A$ oraz $y_B$
\item Alice wyznacza wartość $y = {y_B}^{x_A}$, Bob wyznacza wartość $y = {y_A}^{x_B}$.
\end{enumerate}

\subsection{Bezpieczeństwo}

Wydobycie wartości $x_A$ lub $x_B$ nie jest praktycznie osiągalne z uwagi na problem logarytmu dyskretnego.


\paragraph{Problem Diffiego-Helmana}

Dane $p, g, y_A, y_B$, wyliczamy: $y = g^{x_A x_B} \pmod{p}$.

Podobnie jak z problemem logarytmu dyskretnego, na komputerze klasycznym podejrzewamy, że jest to problem trudny obliczeniowo.

\paragraph{Atak aktywny}

Mallet może przejmować komunikację pomiędzy Alice i Bobem. Przejmuje wysyłane wartości $y_A$ oraz $y_B$, przesyłając wartość $y_M$.
Powoduje to, że Mallet efektywnie stosuje algorytm Diffiego-Hellomana pomiędzy sobą, a Alice oraz sobą, a Bobem.

Atak aktywny czyni algorytm niepraktycznym, z uwagi na to, że Alice i Bob nie mają możliwości potwierdzenia, że wysłana informacja
pochodzi od drugiej oosby, a nie od atakującego.

\section{Protokół ElGamal}

Protokół działa na podstawie dowolnej grupy $(G, \times)$ w której istnieje problem logarytmu dyskretnego (lub izomorficzny).

\subsection{Algorytm szyfrowania z kluczem publicznym.}

Składa się z:

\begin{enumerate}
\item Algorytm generuje klucz tajny $k$ (deszyfrujący) oraz klucz publiczny $K$ (szyfrujący).
Bezpieczeństwo opiera się na trudności obliczeniowej wydobycia klucza tajnego z publicznego.
\item Algorytm szyfrowania
\item Algorytm deszyfrowania
\end{enumerate}


\subsection{Algorytm}

\paragraph{Algorytm generowania klucza}

Alice generuje klucz.

\begin{enumerate}
\item Losuj liczbę pierwszą $p$ (ustalenie grupy $\Phi(p)$.
\item $g \in \Phi(p)$, gdzie $g$ jest generatorem $\Phi(p)$
\item Losuj $1 < x < p$.
\item Oblicz $y = g^x \pmod{p}$
\item Klucz tajny: $k = (x, p)$, klucz publiczny: $K = (p, g, y)$.
\end{enumerate}

\paragraph{Algorytm szyfrowania}

Bob szyfruje wiadomość, będącą $1 < M < p$

\begin{enumerate}
\item Losuje $ 1 < r < p$
\item oblicza $C_1 = g^r \pmod{p}$
\item oblicza $C_2 = My^r \pmod{p}$
\item Pakiet zaszyfrowany to $C = [C_1, C_2]$ i jest on wysyłany do Alice kanałem otwartym.
\end{enumerate}

\paragraph{Algorytm deszyfrowania}
$$
M = C_2 {C_1}^{-x} \pmod{p}
$$

ponieważ $$ C_2 {C_1}^{-x} = My^r(g^r)^{-x} = M g^{xr} g^{-rx} = M \pmod{p}$$

Wrażliwy też jest na problem Diffiego-Helmana, z uwagi na wartości $y = g^x$ oraz $C_1 = g^r$, które zakładając, że da się go rozwiązać, można by było zdeszyfrować wiadomość przez wyliczenie $g^{xr}$.

Dla klucza publicznego $K = (p, g, y)$ i wartości $x$ z klucza prywatnego zastanawiano się nad złożonością obliczania bitów $x = (b_k, \dots, b_1)_2$.
Obliczenie pierwszego bitu $b_1$ jest osiągalne wielomianowo, pozostałe już są problemem trudnym.

Jeśli mamy $a \in \Phi(p)$ i mówimy, że $a$ jest resztą kwadratową jeśli $\exists_b b^2 = a$. Jeśli $x$ jest parzyste, to $y$ jest resztą kwadratową. Jeśli $x$ jest nieparzysty, to $y$ nie jest resztą kwadratową. Sprowadza to rozwiązywanie problemu pierwszego bitu do rozwiązywania reszty kwadratowej.

\subsection{Problem drugiego bitu}

\paragraph{Symbol Lagranga}

Euler udowodnił, że $\left( \frac{a}{p} \right) = a^{\frac{p-1}{2}}$

$$ y = g^{(b_nb_{n-1}\dots b_2 0)_2} $$

Jeśli ostatni bit to zero, to wiemy, że liczba $y$ jest kwadratem - można spierwiastkować. Wychodzi $\sqrt{a} = \pm b$, $\pm b \in \Phi(p)$. $-b$ oraz $b$ są nierożróżnialne gdyż obie są liczbami dodatnimi. Z uwagi na to, nie da się obliczyć bitu $b_2$. Generalnie $b = (b_nb_{n-1}\dots b_2)_2 \lor -b = (b_nb_{n-1}\dots b_2)_2$

Tw. Zakładamy, że istnieje algorytm A, który dla danych $(p, g, y)$ oblicza drugi bit $x$. Wtedy istnieje algorytm $B$, który oblicza $x = log_g(y)$.

\paragraph{Algorytm B}

% b_1 = L_1(y)
% y = yg^{-b_1} \pmod{p}
% i = 1
% while y != 1 do
%   b_i = L_2(y)
%   z = y^\frac{p+1}{4} \pmod{p}
%   if L_1(z) == b_i
%     y = z \pmod{p}
%   else
%     y = -z \pmod{p}
%   g = yg^{-b_i}
%   i = i + 1

Lemat. Niech $p = 3 \pmod{4}$. Dla $y \neq 0$ mamy $L_1(y) \neq L_1(-y)$

\chapter{Wykład 5}

$ E(F_p) $ dla $ p > 2^300 $

\section{Ciało $K$}

Zbiór $(K, +, \times)$ z dwoma działanami, które spełnia następujące warunki:
\begin{itemize}
\item $0$ jest elementem neutralnym operacji addytywnej, $1$ jest elementem neutralnym operacji multiplikatywnej,
\item $(K, +)$ jest grupą abelową,
\item $(K set-difference {0}, \times)$ jest grupą abelową,
\item $\forall a, b, c \in K$ mamy
	\begin{itemize}
		\item $a(b + c) = ab + ac$
		\item $(b + c)a = ba + ca$
	\end{itemize}
\end{itemize}

Liczby wymierne, zespolone i rzeczywiste są ciałami.

$(F_p, +)$ jest równoważne z $Z_p^+$ oraz $(F_p, \times)$ jest równoważne z $\Phi(p)$ dla $p$ będącego liczbą pierwszą.


\section{Krzywa eliptyczna nad $F_p$}

Niech $p > 3$ będzie liczbą pierwszą. Krzywa eliptyczna nad ciałem $F_p$ zdefiniowana jest przez równanie $E: Y^2 = X^3 + AX + B$, gdzie $A, B \in F_p$, a wyróżnik ${\delta}_E = 4A^3 + 27B^2 \neq 0 \pmod{p}$.

Wyróżnik różny od zera dba o brak podwójnych pierwiastków i gwarantuje, że są 3 pierwiastki rzeczywiste dla krzywej.

\subsection{Punkt}

Mówimy, że punkt $P = (x_0, y_0)$ należy do $E/F_p$ jeśli spełnia równanie: $y_0^2 = x_0^3 + Ax_0 + B \pmod{p}$.

Zbiór wszystkich punktów należących do $E/F_p$ to

$$
E(F_p) = \{(x, y) : y^2 = x^3 + Ax + B \pmod{p} \} \cup \mathcal{O}
$$

\paragraph{Przykład} $E : Y^2 = X^3 + 1$. Wyznacz zbiór $E(F_7)$.

\section{Działania w krzywych eliptycznych}

\subsection{Działanie addytywne dla dwóch różnych punktów}

Niech $P, Q \in E(F_p)$, $P = (x_1, y_1)$, $Q = (x_2, y_2)$, $x_1 \neq x_2$.

Wtedy,
$$
	P \oplus Q = R, R = (x_3, y_3)
$$

gdzie

\begin{align*}
	x_3     & = \lambda^2 - x_1 - x_2       & \pmod{p} \\
	y_3     & = \lambda (x_1 - x_3) - y_1   & \pmod{p} \\
	\lambda & = (y_2 - y_1)(x_2 - x_1)^{-1} & \pmod{p}
\end{align*}

\subsection{Działanie addytywne dla takiego samego punktu}

$$ P \oplus P = Q $$

gdzie

\begin{align*}
	x_3     & = \lambda^2 - 2x_1         & \pmod{p} \\
	y_3     & = \lambda(x_1 - x_3) - y_1 & \pmod{p} \\
	\lambda & = (3x_1^2 + A)(2y1)^{-1}   & \pmod{p}
\end{align*}

a A to współczynnik z równania krzywej.

Działanie to jest łączne oraz przemienne.

\subsection{Punkt przeciwny}

Przyjmujemy, że jeśli $P = (x, y)$, to $-P  = (x, -y)$. Przyjmujemy, że $P \oplus (-Q) = P \ominus Q$

\paragraph{Element neutralny}

$$P \oplus -P = \mathcal{O}$$

\subsection{Liczba punktów krzywej}

$$
\#E(F_p) = p + 1 - t, |t| \leq 2\sqrt{p}
$$

Jeśli krzywa ma rząd, który jest liczbą pierwszą to wystarczy by $p$ było rzędu 300 bitów by był bezpieczny.

\section{Protokół Diffiego-Helmana}

\begin{enumerate}
	\item Alice losuje $p$ i $E/F_p$ taką, że $\#E(F_p)$ jest liczbą pierwszą
\end{enumerate}

Lista fajnych krzywych: https://safecurves.cr.yp.to/

\end{document}
